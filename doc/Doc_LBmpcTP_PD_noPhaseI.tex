\documentclass[letter]{article}
\addtolength{\textwidth}{4cm}
\addtolength{\textheight}{3cm}
\addtolength{\hoffset}{-2.5cm}
\addtolength{\topmargin}{-1cm}
\usepackage[pdftex]{graphicx}
\usepackage[english]{babel}
\usepackage[latin9]{inputenc}
\usepackage{amsmath}
\usepackage{amssymb}
\usepackage{amsfonts}		%mathematisches Menge R, N
\usepackage[usenames]{color}
%\usepackage{textcomp}
%\usepackage{latexsym}		%für Re(z) und Img(z)
%\usepackage{trfsigns}		%für Fouriertrafos
\usepackage{mathrsfs}
\usepackage[hyperindex,pdfpagelabels=true,plainpages=false]{hyperref}
\usepackage[square, comma, sort&compress, numbers]{natbib}
\usepackage{listings}
%\usepackage{pstricks}
\usepackage[numbered]{mcode}
\usepackage{url}


\date{\today}
\author{Xiaojing Zhang \\ \\
 	{\bf Documentation of \texttt{LBmpcTP} template class} \\
 	{\it Implementation of primal-dual infeasible start interior point method (PD IIPM)} \\
 	 Department of Electrical Engineering and Computer Sciences (EECS), UC Berkeley}
\title{Learning Based MPC}

%Vorspann
\begin{document}
\pagestyle{headings}	%Beschriftungen oben
\maketitle

\begin{sffamily}		%alles Serifenlos

%\newpage

\newcommand{\sinc}{\mathrm{sinc}}
\renewcommand{\thefootnote}{\fnsymbol{footnote}}

\section*{Introduction}
This report introduces the \texttt{LBmpcTP} template class which implements the primal-dual infeasible interior point method (\textbf{PD IIPM}) based on Mehrotra's predictor-corrector algorithm \citep{Mehrotra1992PDIPM}. Our solver is tailored to  learning-based model predictive control (LBMPC) framework \citep{Aswani2011LBMPC} for the special case where the costs are quadratic and the involved dynamics affine.\\

\noindent
The report is structured as follows: First, LBMPC for the case where all dynamics are affine and quadratic cost is introduced. Second, the interface to the \texttt{LBmpcTP} class is presented. In the third section, advanced parameters for configuration and fine-tuning are described.

\section{The Learning-Based MPC Model}
We consider a special case of LBMPC \citep{Aswani2011LBMPC}, in which all dynamics are linear and the cost quadratic. Furthermore, we consider the situation where the feasible sets are convex polyhedron. \\

\noindent
This instance of LBMPC is given by the following optimization problem:
\begin{align}\label{eq: LBMPC}
	\min_{c[\cdot],\theta} \quad & (\tilde{x}[m+N|m]-x^\star[m+N|m])^T \tilde{Q}_f (\tilde{x}[m+N|m]-x^\star[m+N|m]) + \\
	&  \sum_{i=0} ^ {N-1} \{ (\tilde{x}[m+i|m]-x^\star[m+i|m])^T \tilde{Q} (\tilde{x}[m+i|m]-x^\star[m+i|m]) + \nonumber \\
	& (\check{u}[m+i|m]-u^\star[m+i|m])^T R (\check{u}[m+i|m]-u^\star[m+i|m]) \}  \nonumber \\	
		\nonumber \\		
	\text{s.t.} \quad & \tilde{x}[m|m] = \hat{x}[m],\quad \bar{x}[m|m] = \hat{x}[m] \nonumber \\
		&  \tilde{x}[m+i|m]= A \tilde{x}[m+i-1|m] + B \check{u}[m+i-1|m] + s + \mathcal{O}_m(\tilde{x}[m+i-1|m],\check{u}[m+i-1|m]), \nonumber\\
		& \mathcal{O}_m(\tilde{x}[m+i-1|m],\check{u}[m+i-1|m]) = L_m \tilde{x}[m+i-1|m] + M_m \check{u}[m+i-1|m] + t_m,  \nonumber \\
	 &	\bar{x}[m+i|m]= A \bar{x}[m+i-1|m] + B \check{u}[m+i-1|m]+s, \nonumber\\
	 & \check{u}[m+i-1|m] = K\bar{x}[m+i-1|m] + c[m+i-1|m], \nonumber \\
	 & F_{\bar{x},i}\bar{x}[m+i|m] \leq f_{\bar{x},i}, \quad F_{\check{u},i}\check{u}[m+i|m] \leq f_{\check{u},i}, \nonumber \\
	 & F_{x\theta}\bar{x}[m+j|m] + F_\theta \theta \leq f_{x\theta},\quad \text{for some } j\in\{1,\ldots, N\} \nonumber
\end{align}

\noindent
We assume that the conditions given in Table \ref{tab: matrix assumptions} hold. \\

\begin{table}[!htdp]
\caption{We make the following assumptions on the matrices in \eqref{eq: LBMPC}.}
\begin{center}
\begin{tabular}{|l|l|}\hline
 $\tilde{Q}\in\mathbb{R}^{n\times n}$ & positive definite\\ 
 $\tilde{Q}_f\in\mathbb{R}^{n\times n}$ & positive definite\\ 
 $R\in\mathbb{R}^{m\times m}$ & positive semidefinite \\ \hline
 $F_{\bar{x},i}\in\mathbb{R}^{\texttt{\_nSt}\times n}\quad \forall i$ & full rank \\ 
 $F_{\check{u},i}\in\mathbb{R}^{\texttt{\_nInp}\times m}\quad \forall i$ & full rank \\ 
 $F_\theta\in\mathbb{R}^{\texttt{\_nF\_xTheta}\times m}$ & full rank \\ \hline
 $K\in\mathbb{R}^{m\times n}$ & $A+BK$ Schur \\ \hline
\end{tabular}
\end{center}
\label{tab: matrix assumptions}
\end{table}


\noindent
For the purpose of solving this LBMPC problem, it is useful to consider the optimization problem at each time step as belonging to a general class of problems, in this case a convex quadratic program (QP-LBMPC):

\begin{align}\label{eq: reformulation}
	 \min_{z} \qquad & z_m^T H z_m + g_m^T z_m  \\
	  \text{s.t.} \qquad & C_m z_m = b_m \nonumber \\
	 & P z_m \leq h_m, \nonumber 
\end{align}

\noindent
where $z$ is the stacked vector:

\begin{equation*}
	z_m = \begin{pmatrix}	
		c[m]^T & \tilde{x}[m+1]^T & \bar{x}[m+1]^T & \cdots & c[m+N-1]^T & \tilde{x}[m+N]^T & \bar{x}[m+N]^T & \theta^T	\end{pmatrix}^T.
\end{equation*}
For the sake of clarity, we omit the index $m$ in \eqref{eq: reformulation} in all following calculations.\\

\noindent
It can be shown that the above QP-LBMPC is convex. Furthermore, we assume that Slater's condition (or any other constraint qualification)  holds \citep[Ch. 2]{Borrelli2011Book}. Hence, the KKT-conditions for optimality are necessary and sufficient. They state that for every optimal $z_\text{opt}$ there exists vectors $\lambda_\text{opt}, \nu_\text{opt}$ and $t_\text{opt}$ such that at the optimal point $(z,\lambda,\nu,t)=(z_\text{opt},\lambda_\text{opt},\nu_\text{opt},t_\text{opt})$ the following equations are satisfied \citep{Nocedal2000NumOpt}:
\begin{equation}
\label{eq: KKT}
	\mathcal{F}(z,\lambda,\nu,t) \triangleq 
	\begin{pmatrix}
		r_H \\ r_C \\ r_P \\ r_T
	\end{pmatrix} \triangleq
	\begin{pmatrix}
		2Hz + g + P^T \lambda + C^T \nu \\
		Cz - b \\
		Pz-h+t \\
		T\Lambda\mathbf{1}
	\end{pmatrix} = 0,\qquad (\lambda, t) \geq 0
\end{equation}
where $t$ is the slack variable associated with the inequality in \eqref{eq: reformulation}, $T\triangleq\text{diag}(t),\ \Lambda\triangleq\text{diag}(\lambda)$ and $\mathbf{1}$ is the all-one vector. Our PD IIPM algorithm generates sequences $(z^i,\lambda^i,\nu^i,t^i)$ with $(\lambda^i,t^i)>0$ that approach the optimality condition \eqref{eq: KKT}. \\

\noindent
Because the PD IIPM algorithm is not guaranteed to generate feasible iterates (except in the limit as the algorithm converges), a duality gap cannot be defined. Instead, the complementary measure $\mu$ is used to measure the optimality of the point $(z,\lambda,\nu,t)$:
\begin{equation}
\label{eq: dualityGap}
	\mu\triangleq \frac{\lambda^T t}{m_P},
\end{equation}
where $m_P$ is the number of inequality equations, i.e. the number of rows in the Matrix $P$.

\section{Using the \texttt{LBmpcTP} template class}
The \texttt{LBmpcTP} template class is typically called in two seperate steps:

\begin{enumerate}
	\item Definition of matrices $A,\ B,\ s,\ \tilde{Q},\ \tilde{Q}_f,\ R,\ K,\ \{F_{\bar{x},i}\}_{i=1}^N,\ \{f_{\bar{x},i}\}_{i=1}^N,\ \{F_{\check{u},i}\}_{i=0}^{N-1},\ \{f_{\check{u},i}\}_{i=0}^{N-1},\ F_{x\theta},\ F_\theta,$ $f_{x\theta}$, scalars $n_\text{iter},\ \epsilon_\text{reg},\ \epsilon_\text{primal},\ \epsilon_\text{dual},\ \epsilon_\mu$ and string \texttt{fileName} in MATLAB file \texttt{Init.m}, see Table \ref{tab: parameters Init.m}.
	\item The main C++-file (e.g. \texttt{mainLBmpcTP.cpp}) instantiates \texttt{LBmpcTP}-object and calls the solver routine, i.e.:
	\begin{enumerate}
		\item Instantiation of an object, e.g. \texttt{myObj}: \\
		\texttt{LBmpcTP<double, \_n, \_m, \_N, \_nSt, \_nInp, \_nF\_xTheta, \_pos\_omega> myObj( fileName, verbose)}
		\item It calls the step-function \texttt{myObj.step(.)} which computes the optimal input and returns a status flag. Each call of \texttt{myObj.step} requires the following set of (updated) parameters: $L_m$, $M_m$, $t_m$, $\hat{x}$, $\{x^\star[m+i]\}_i$. \\
		\texttt{status = myObj.step( Lm, Mm, tm, x\_hat, x\_star );}\\
		\texttt{u\_opt = myObj.u\_opt;}
	\end{enumerate}
\end{enumerate}

\noindent
The \texttt{C++} code is then compiled using a suitable compiler. In case of the gcc-compiler, the compilation may look like this: \newline
\texttt{g++ -I /usr/local/include/eigen3/ -O3 mainLBmpcTP.cpp -o mainLBmpcTP}. \\


\noindent
In the following sections, steps 1. and 2. are described in more detail.

\subsection{MATLAB: \texttt{Init.m}}
\texttt{Init.m} defines the parameters required for the instantiation of the \texttt{LBmpcTP} object and consists of the following two parts:
\begin{itemize}
	\item User specifies the parameters given in Table \ref{tab: parameters Init.m}.
	\item The MATLAB script \texttt{writeParam.m} writes the parameters to the binary file specified by \texttt{fileName} (default: \texttt{ConstrParam.bin}).
\end{itemize}
Appendix \ref{app: Init.m} shows a typical implementation of \texttt{Init.m}. \\

\begin{table}[!htdp]
\caption{Key parameters which are to be defined in \texttt{Init.m}}
\begin{center}
\begin{tabular}{|l|l|l|}\hline
{\bf MATLAB variable} & description & typical range/value \\ \hline\hline

 \texttt{N} $\in\mathbb{N}$ & length of MPC horizon & \\ 
 \texttt{m} $\in\mathbb{N}$ & number of inputs & \\ 
 \texttt{n} $\in\mathbb{N}$ & number of states & \\  \hline
 
 \texttt{fileName} & name of binary file that stores the matrices and scalars below. & \\  \hline
 
 \texttt{A} $\in\mathbb{R}^{n\times n}$ & linear dynamics matrix: $\bar{x}^+ = A\bar{x}+B\check{u}+s$ & \\ 
 \texttt{B} $\in\mathbb{R}^{n\times m}$ & input-state dynamics matrix: $\bar{x}^+ = A\bar{x}+B\check{u}+s$ & \\ 
 \texttt{s} $\in\mathbb{R}^{n}$ & affine offset in state dynamics:  $\bar{x}^+ = A\bar{x}+B\check{u}+s$ & \\ 
 \texttt{K} $\in\mathbb{R}^{m\times n}$  & feedback gain matrix, $\check{u}=K\bar{x}+c$, $A+BK$ is stable & \\  \hline
  
 \texttt{Q\_tilde} $\in\mathbb{R}^{n\times n}$ & p.d. weight matrix for state & \\ 
 \texttt{Q\_tilde\_f} $\in\mathbb{R}^{n\times n}$ & p.d. weight matrix for final state & \\ 
 \texttt{R} $\in\mathbb{R}^{m\times m}$ & p.s.d. weight matrix on input & \\ \hline
 
 \texttt{Fx\{$i$\}} $\in\mathbb{R}^{\texttt{\_nSt}\times n}$ & $F_{\bar{x},i}\bar{x}[m+i|m] \leq f_{\bar{x},i}$, full-rank, $i=1,\ldots,N$ & \\ 
 \texttt{fx\{$i$\}} $\in\mathbb{R}^{\texttt{\_nSt}}$ & $F_{\bar{x},i}\bar{x}[m+i|m] \leq f_{\bar{x},i}$, $i=1,\ldots,N$ & \\ 
 \texttt{Fu\{$i$\}} $\in\mathbb{R}^{\texttt{\_nInp}\times m}$ & $F_{\check{u},i}\check{u}[m+i|m] \leq f_{\check{u},i}$, full-rank, $i=0,\ldots,N-1$ & \\ 
 \texttt{fu\{$i$\}} $\in\mathbb{R}^{\texttt{\_nInp}}$ & $F_{\check{u},i}\check{u}[m+i|m] \leq f_{\check{u},i}$, $i=0,\ldots,N-1$ & \\ 
 \texttt{F\_xTheta} $\in\mathbb{R}^{\texttt{\_nF\_xTheta}\times n}$ & $F_{x\theta}\bar{x}[m+j|m] + F_\theta \theta \leq f_{x\theta}$, $j\in\{1,\ldots,N\}$ & \\ 
 \texttt{F\_theta} $\in\mathbb{R}^{\texttt{\_nF\_xTheta}\times m}$ & $F_{x\theta}\bar{x}[m+j|m] + F_\theta \theta \leq f_{x\theta}$, full-rank, $j\in\{1,\ldots,N\}$ & \\ 
 \texttt{f\_xTheta} $\in\mathbb{R}^{\texttt{\_nF\_xTheta}}$ & $F_{x\theta}\bar{x}[m+j|m] + F_\theta \theta \leq f_{x\theta}$, $j\in\{1,\ldots,N\}$ & \\ \hline 

 \texttt{n\_iter} $\in\mathbb{N}$ & max. number of Newton steps to solve \eqref{eq: reformulation} & $100$ \\ 
 \texttt{reg} $\in\mathbb{R}$  & regularization coefficient to render Matrix $H$ \eqref{eq: reformulation} positive definite & $[0,0.1]$, depends on $H$ \\ 
 \texttt{eps\_primal}  $\in\mathbb{R}$& $\epsilon_\text{primal}$, necessary stopping criteria, see \eqref{eq: stoppingCriteria} & $0.1$ \\ 
 \texttt{eps\_dual} $\in\mathbb{R}$ & $\epsilon_\text{dual}$, necessary stopping criteria, see \eqref{eq: stoppingCriteria} & $0.1$ \\ 
 \texttt{eps\_mu} $\in\mathbb{R}$ & $\epsilon_\mu$, necessary stopping criteria, see \eqref{eq: stoppingCriteria} & $0.1$ \\ \hline
\end{tabular}
\end{center}
\label{tab: parameters Init.m}
\end{table}

\noindent
Remarks:
\begin{itemize}
	\item The number of state constraints is assumed to be constant, i.e. the number of rows in \texttt{Fx\{$i$\}} is constant for all $i$, and  denoted by \texttt{\_nSt}.
	\item The number of input constraints is assumed to be constant, i.e. the number of rows in \texttt{Fu\{$i$\}} is constant for all $i$, and denoted by \texttt{\_nInp}.
	\item The number of constraints with $\theta$ in \eqref{eq: LBMPC} is assumed to be \texttt{\_nF\_xTheta}.
\end{itemize}


\subsection{C++: \texttt{mainLBmpcTP.cpp}}
This file is the main control routine that manipulates \texttt{LBmpcTP} objects. An example file is provided in Appendix \ref{app: mainLBmpcTP.cpp}. In the following, we outline the main structure of \texttt{mainLBmpcTP.cpp}:

\begin{enumerate}
	\item \textbf{SPECIFY parameters:} \texttt{\_N}, \texttt{\_m}, \texttt{\_n}, \texttt{\_nSt}, \texttt{\_nInp}, \texttt{\_nF\_xTheta}, \texttt{\_pos\_omega} (see Table \ref{tab: instantiateLBpmcTP}).
	
		\begin{table}[!htdp]
		\caption{The following template parameters are required to instantiate a \texttt{LBmpcTP}-object.}
		\begin{center}
		\begin{tabular}{|l|l|l|}\hline
		{
 			\bf variable} & description & default \\ \hline \hline
 
			\texttt{Type} & only \texttt{double} is supported & \texttt{double} \\ \hline
 			\texttt{\_N} & length of MPC horizon & \\ 
 			\texttt{\_m} & number of inputs & \\ 
 			\texttt{\_n} & number of states & \\ \hline 
 			\texttt{\_nSt} & number of state constraints (constant over the horizon) & \\ 
 			\texttt{\_nInp} & number of input constraints (constant over the horizon) & \\ 
	 		\texttt{\_nF\_xTheta} & number of constraints involving $\theta$ in \eqref{eq: LBMPC} & \\ \hline
 			\texttt{\_pos\_Omega} & index $j\in\{1,\ldots,N\}$ in $F_{x\theta}\bar{x}[m+j|m] + F_\theta \theta \leq f_{x\theta}$ & \\ \hline 
		\end{tabular}
		\end{center}
		\label{tab: instantiateLBpmcTP}
		\end{table}	
	
	\item \textbf{SPECIFY binary source file name and verbose-flag:} \texttt{fileName}, \texttt{verbose}: \\
	\texttt{const char fileName[] = "ConstrParam.bin";} \\
	\texttt{bool verbose = 0;\qquad // 0 = shut up}
	
	\item Call the constructor and instantiate an object (e.g. \texttt{myObj}): \\
	\texttt{LBmpcTP<double,\_n,\_m,\_N,\_nSt,\_nInp,\_nF\_xTheta,\_pos\_omega> myObj(fileName,verbose);}
	
	\item Update the variables needed for step()-function: \texttt{Lm}, \texttt{Mm}, \texttt{tm},  \texttt{x\_hat}, \texttt{x\_star[\_N]}  (Table \ref{tab: oracleParameters}). Note: these values are not provided by \texttt{LBmpcTP}.
	
	\begin{table}[!htdp]
	\caption{These arguments must be updated before each \texttt{step()}-call.}
	\begin{center}
	\begin{tabular}{|l|l|}\hline
	{
 		\bf variable} & description \\ \hline\hline
 
		 \texttt{Lm} & oracle matrix, i.e. $\mathcal{O}_m(\tilde{x}[m+i|m],\check{u}[m+i|m]) = L_m \tilde{x}[m+i|m] + M_m \check{u}[m+i|m] + t_m$ 					 \\ 
 		\texttt{Mm} & oracle matrix, i.e. $\mathcal{O}_m(\tilde{x}[m+i|m],\check{u}[m+i|m]) = L_m \tilde{x}[m+i|m] + M_m \check{u}[m+i|m] + t_m$ 					 \\ 
 		\texttt{tm} & oracle matrix, i.e. $\mathcal{O}_m(\tilde{x}[m+i|m],\check{u}[m+i|m]) = L_m \tilde{x}[m+i|m] + M_m \check{u}[m+i|m] + t_m$				\\ \hline
 		\texttt{x\_hat} & current state estimate, i.e. $\tilde{x}[m] = \hat{x}[m],\quad \bar{x}[m] = \hat{x}[m]$ \\ \hline
 		\texttt{x\_star[\_N]} & states our system wants to track \\ \hline
	\end{tabular}
	\end{center}
	\label{tab: oracleParameters}
	\end{table}	
	
	\item Call the step-function to solve the optimization problem: \\
	\texttt{status = myObj.step( Lm, Mm, tm, x\_hat, x\_star );} \\
	Table \ref{tab: error code} lists the possible status codes together with their meaning.
	
	\begin{table}[!htdp]
	\caption{The meaning of the status-flags returned by \texttt{step()}-function.}
	\begin{center}
	\begin{tabular}{|l|l|}\hline
	{
	 \bf status flag} & meaning  \\ \hline\hline
	 0 & success \\
	 1 & problem (possibly) primal infeasible\\
	 2 & problem (possibly) dual infeasible \\
	 3 & stopping criterion $\mu \leq \epsilon_\mu$ not satisfied \\
	 4 & \texttt{nan} \\
	 5 & other error \\ \hline
	\end{tabular}
	\end{center}
	\label{tab: error code}
	\end{table}
	
	\item The optimal input can be accessed by: \texttt{u\_opt = myObj.u\_opt;}
\end{enumerate}


\noindent
Note that the parameters in \texttt{Init.m} and \texttt{mainLBmpcTP.cpp} have to be consistent with each other.



\subsection{C++ template class: \texttt{LBmpcTP.h}}
Here, we give a rough overview of \texttt{LBmpcTP}'s internal structure. As mentioned, basic access to the class is granted through two methods (constructor and \texttt{step(.)}-method) as well as the public member variable \texttt{u\_opt}. Details on the implementation and underlying mathematics can be found in \citep{Zhang2012fastLBMPCPaper, Zhang2011fastLBMPC} \citep{Mehrotra1992PDIPM, Potra2000IPM, Rao98IMP2MPC, Nocedal2000NumOpt}. The constructor initializes the private variables discussed in the previous sections. The \texttt{step(.)}-method performs the following tasks:
\begin{itemize}
	\item It recursively computes the sequence $\{  u^\star[m+i|m]  \}_i$ from the given desired state sequence $\{ x^\star[m+i|m] \}_i$ by solving 
	\begin{equation*}
			x^\star [m+i|m] = (A+L_m)x^\star[m+i-1|m] + (B+M_m)u^\star[m+i-1|m] + (s+t_m),\quad x^\star[m|m]=\hat{x}
	\end{equation*}
	and taking the least-squares solution (SVD).
	\item Casts \eqref{eq: LBMPC} into \eqref{eq: reformulation}.
	\item Finally, it computes the optimal control input and stores it in the public member variable \texttt{u\_opt}.
\end{itemize}


\section{Advanced Topics}
Some parameters in Tab. \ref{tab: parameters Init.m} can be used to tweak \texttt{LBmpcTP}:
\begin{itemize}
	\item The problem cannot be solved with the default parameters. It either does not converge (number of iterations exceeds \texttt{num\_iter}) or the program returns \texttt{nan} (i.e. error code $4$).
	\item Convergence is too slow for the desired purpose, i.e. the optimization step needs too many Newton iterations.
	\item The exact solution is not necessary and an approximate solution suffices to speed up algorithm.
\end{itemize}
The goal of this section is to describe some implementation details.\\

\noindent
Table \ref{tab: tuning parameters} lists the tuning parameters from Tab. \ref{tab: parameters Init.m} and describes their role and influence in greater detail.

\begin{table}[!htdp]
\caption{Tuning parameters defined in \texttt{Init.m}}
\begin{center}
\begin{tabular}{|p{3.5cm}|p{10cm}|p{3cm}|}\hline
{
	\bf tuning variable} & influence & typical range/value \\ \hline\hline

  \texttt{n\_iter} & Can be used to limit the number of Newton iterations or for early termination to  obtain an inaccurate solution of \eqref{eq: reformulation}. This can be useful if the computational time is limited & $100$ \\ \hline
 \texttt{reg} & regularization coefficient to render Matrix $H$ in \eqref{eq: reformulation} positive definite, see \eqref{eq: regularization}  & $[0,0.1]$, depends on $H$ \\ \hline
 \texttt{eps\_primal} & $\epsilon_\text{primal}$, necessary stopping criteria, see \eqref{eq: stoppingCriteria} & $0.1$ \\ \hline
 \texttt{eps\_dual} & $\epsilon_\text{dual}$, necessary stopping criteria, see \eqref{eq: stoppingCriteria} & $0.1$ \\ \hline
 \texttt{eps\_mu} & $\epsilon_\mu$, necessary stopping criteria on $\mu$, see \eqref{eq: stoppingCriteria} & $0.1$ \\ \hline
\end{tabular}
\end{center}
\label{tab: tuning parameters}
\end{table}


\subsection{Compiling}
The EIGEN\footnote{\url{http://eigen.tuxfamily.org/}} library has to be installed to use \texttt{LBmpcTP}.\\

\noindent
Furthermore, some compilers provide the option to generate optimized executable codes. For example, the gcc compiler allows the user to add the \texttt{-O3} option which  reduces the size of the executable file and increases the performance of the generated code: \\
\noindent
\texttt{g++ -I /usr/local/include/eigen3/ -O3 mainLBmpcTP.cpp -o mainLBmpcTP} \\

\noindent
Also, it is advised to use the latest compiler for compatibility and performance reasons. \texttt{LBmpcTP} is tested to work with gcc versions 4.2 and 4.6.

\subsection{Stopping Criterion and Regularization}
\label{sec: regularization}
This section addresses the termination criterion. \texttt{LBmpcTP} uses a heuristic stopping criterion adapted from \citep{Gertz2003OOQP, VanderbreiLOQO} and consists of the following three conditions:
\begin{align}
\label{eq: stoppingCriteria}
	\frac{\| \begin{pmatrix}   r_C^T & r_P^T  \end{pmatrix}^T \|}{ \| \begin{pmatrix} h^T & b^T \end{pmatrix}^T \|+1    } &\leq \epsilon_\text{primal} \\
	\frac{\|r_H\|}{\|g\|+1} &\leq \epsilon_\text{dual}	\nonumber  \\
	\mu &\leq \epsilon_\mu \nonumber.
\end{align}
The algorithm terminates successfully if and only if all three conditions are satisfied. It should be mentioned at this place that the smaller we choose $\epsilon_{\text{primal}},\epsilon_{\text{dual}},\epsilon_{\mu}$ to be, the ill-conditioned our problem becomes. Because the quadratic cost matrix $H$ in \eqref{eq: LBMPC} is not strictly convex, numerical issues arise as we try to push the residuals \eqref{eq: KKT} towards zero: the linear equation becomes poorly-conditioned, posing challenges when computing the Cholesky decomposition numerically. This issue has been widely discussed, e.g. in \citep{Wright1997StabAugSystemFact, Wright1995StabilityLinEqSolversIPM}. In \texttt{LBmpcTP}, the regularization parameter $\epsilon_\text{reg}$ (\texttt{reg} in Table \ref{tab: parameters Init.m}) is introduced that regularizes the coefficient matrix in the so-called "augmented form", i.e. instead of solving a linear equation with the coefficient matrix
\begin{equation}
\label{eq: augmentedForm}
\begin{pmatrix}
\Phi & C^T \\
		C & 0
	\end{pmatrix},
\end{equation}
where $\Phi>0$ is ill-conditioned, we solve with the regularized coefficient matrix
\begin{equation}
\label{eq: regularization}
	\begin{pmatrix}
		\Phi & C^T \\
		C & 0
	\end{pmatrix} + 
	\epsilon_\text{reg}
	\begin{pmatrix}
		I & 0 \\ 0 & 0
	\end{pmatrix}
\end{equation}
where $\mathbb{I}$ is the identity matrix. As $\epsilon\to0$, the solution of \eqref{eq: regularization} approaches the solution of \eqref{eq: augmentedForm} because $\Phi>0$.

\subsection{Troubleshooting}
In this section, some common errors are described. Possible sources for these errors are given and solutions are proposed. We assume that the posed problem has a solution, i.e. that it is primal and dual feasible.
\begin{enumerate}	
	\item \textbf{We only want to approximately solve \eqref{eq: reformulation}.} \\
		\emph{Solutions:}
			\begin{itemize}				
				\item This can be achieved by bounding the permitted number of Newton iterations by choosing a small \texttt{n\_iter}. Depending on the problem, numbers as few as $3$ iterations are enough to produce satisfying results.
			\end{itemize}
	
	\item \textbf{Obtained result is a \texttt{nan}-vector (not a number).} \\
		\emph{Solutions:}
		\begin{itemize}
			\item This problem is directly related to the ill-conditioning described in Section \ref{sec: regularization}. The way to handle this in \texttt{LBmpcTP} is by choosing $\epsilon_\text{reg}$ such that the program does not produce \texttt{nan} (not a number).
		\end{itemize}
\end{enumerate}

\subsection{Additional Remarks}
\begin{itemize}
	\item So far, the algorithm only works for a minimum prediction horizon of $3$.
	\item There are more parameters and class methods in \texttt{LBmpcTP.h} that can be tuned to improve the performance of the solver. These include
	\begin{itemize}
		\item How to implement the regularization in class method \texttt{compPhi()}.
		\item Initialization of starting point $(z^0,\lambda^0,\nu^0,t^0)$ using a different heuristic in \texttt{compInitPoints()}.
		\item How to choose the final step size: 
		\begin{enumerate}
			\item Use Mehrotra's heuristic as it is done in  class method \texttt{compAlpha\_corrector()} and choose the parameter $0<\texttt{gamma\_f}\ll 1$.
			\item Do it using method \texttt{compAlpha\_affine()} and some predefined damping factor $0<\texttt{damp}\ll 1$
		\end{enumerate}
		\item Use another kind of stopping criterion.
		\item Use another heuristic to detect primal and dual infeasibility.
	\end{itemize}		
 However, it usually suffices to tune  the parameters given in Table \eqref{tab: tuning parameters}.
 
	\item If the prediction horizon $N$ is large, then the variable \texttt{Nhoriz} in the class file \texttt{LBmpcTP.h} has to be increased. The reason is that some arrays are preallocated with length \texttt{Nhoriz}, which is due to a limitation in EIGEN.
	
	\item When an \texttt{LBmpcTP} object is instantiated, the class variable \texttt{z} is initialized. Between the time steps, \texttt{z} can take  the role of "warm start". However, when the \texttt{LBmpcTP} is instantiated, it set as the $0$-vector. If a priori information is available, then a more suitable \texttt{z} can be chosen.
\end{itemize}



\newpage
\section{Example Files}
\subsection{\texttt{Init.m}} \label{app: Init.m}
\lstset{language=Matlab}
\lstinputlisting{Init.m}

\subsection{\texttt{writeParam.m}} \label{app: writeParam.m}
\lstset{language=Matlab}
\lstinputlisting{Init.m}

\newpage
\subsection{ \texttt{mainLBmpcTP.cpp}} \label{app: mainLBmpcTP.cpp}
\lstset{language=C}
\lstinputlisting{mainLBmpcTP.cpp}



\end{sffamily}		%Ende alles Serifenlos

\bibliographystyle{IEEEtran}
\bibliography{bibliography}


\end{document}






